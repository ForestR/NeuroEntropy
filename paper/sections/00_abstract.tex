\begin{abstract}
We present a theoretical and empirical investigation of structural vulnerabilities in large language models, demonstrating that larger models exhibit increased fragility to Hessian-aware eigen-perturbations that induce spectral collapse. Our work establishes an inverse scaling law: under identical optimization conditions, vulnerability to null-space adversarial perturbations increases monotonically with model size. We show that rank reduction scales from 2.98\% (70M) to 4.07\% (160M) to 5.82\% (410M) to 7.65\% (1B), with the 1B model suffering the most severe effective rank reduction. Critically, we demonstrate that quantization acts as a discrete gradient barrier, with 8-bit quantization reducing attack effectiveness by approximately 14\times compared to FP16. Furthermore, we demonstrate that the collapsed state exhibits severe numerical instability, rendering standard recovery via supervised fine-tuning intractable due to floating-point overflow in forward passes. These findings have implications for AI safety and model architecture design.
\end{abstract}
